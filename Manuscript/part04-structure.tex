

\documentclass[main.tex]{subfiles}

\begin{document}

\chapter{Cross section characterization}

\begin{itemize}
\item Strength envelope for combined loading
\item Effect of the bond law on cross-section behavior
\item Where can it be used as well?
\item Limits of validity – what happens if the scales of process zones overlap
\end{itemize}

\chapter{Beams and Frames}

\begin{itemize}
\item Numerical treatment of crack development and debonding
\item Shear zones 
\end{itemize}

\chapter{Plates/slabs}

- Application of dimensioning and assessment tool

\chapter{Shells}
- Form follows force


\chapter{Volume}
- Foundations (RC)



\chapter{Folded structures}



\chapter{Dimensioning approaches}


\section{Index of referenced topics}

\begin{itemize}
\item What types of materials can be combined in a meaningful way?
 \item What combinations of matrices and reinforcement make sense.
 \item What kind of stress-strain response can be achieved?
 \item Can a composite consisting of ductile components exhibit ductile behavior?
 \item What is the fiber-bundle analogy?
 \item Heterogeneous matrix, heterogeneous reinforcement.
\item Secondary debonding effects - splitting, Heuer effekt
\item Multiple cracking
\item Damage-induced anisotropy
\item What is the correspondence between damage-based description of softening behavior and fracture mechanics based description.
\item What is the effect of material heterogeneity on the material behavior?
\item Matrix crack initiation, propagation, interaction
\item Reinforcement debonding, anchorage, ideal uniaxial bond, profiled bond
\item{What is the effect of scatter?}
\end{itemize}


\end{document}